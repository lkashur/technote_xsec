\documentclass[../main.tex]{subfiles}
\graphicspath{{\subfix{../figures/}}}
\begin{document}

\section{Systematic Uncertainties}
\label{sec:systs}
The systematic uncertanties relevant to this analysis are divided into three categories: (1) beam flux, (2) interaction modeling, and (3) detector response modeling.  In this section, the methods for addressing each uncertainty are discussed.

\subsection{Flux Uncertainties}
The BNB flux at ICARUS is based on predictions from the MiniBooNE collaboration, which used the Geant4 simulation tool kit to model the propagation of particles produced in proton beam-target interactions \cite{bnbflux}.  Treatment of flux uncertainties, which arise from sources like hadron production, hadronic secondary interactions within the beam target, and beam focusing, are handled with a many-universes technique.  In this approach, uncertainties are determined by varying the underlying parameters from each source simultaneously in a set of ``universes'' to construct a covariance matrix.  The full set of parameters and the uncertainties they are varied within are shown in Table \ref{Tab:fluxparameters}.

\begin{table}[ht]
    \caption{Purity and efficiency for $\nu_{\mu}$ CC $\pi^{0}$ Selection Cuts}
    \vspace{0.1cm}
    \centering
    \begin{tabular}{ c c c } 
    \hline
    Parameter & Uncertainty & Other \\
    \hline
    pi production & 0\% & 0 \\ 
    pi production & 0\% & 2.64 \\
    K production & 0\% & 4.03 \\
    K production & 0\% & 83.18 \\
    \hline
    \end{tabular}
    \label{Tab:fluxparameters}
\end{table}

\subsection{Cross Section Uncertainties}
\subsection{Detector Uncertainties}

\end{document}